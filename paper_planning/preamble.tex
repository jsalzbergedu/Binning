% This is stuff that is required for the document to compile. It is not
% related to formatting. 
%\usepackage[pdftex]{graphicx}
% xcolor clashes with osdi20 template
%\usepackage[dvipsnames]{xcolor} % for Blue; also, must before tikz, pstricks
\usepackage{booktabs}  % for tables
\usepackage{dcolumn}   % for tables
%\usepackage{hhline}   % for tables
\usepackage{multirow}  % for tables
\usepackage{rotating}
\usepackage{xspace}
\usepackage{inconsolata} % nicer monospace font
\usepackage{hyphenat}  % supplies \hyp{}, which tells tex that it can 
		       % hyphenate at an existing hyphen
%\usepackage[table,usenames,dvipsnames]{xcolor}
\usepackage{enumerate}
\usepackage{mfirstuc} % captialize first letter
\usepackage{tikz} % two-way onarrow
\usepackage{xparse} % two-way onarrow
% for epoch transaction
\usepackage{listings}
\definecolor{dkgreen}{rgb}{0,.6,0}
\definecolor{dkblue}{rgb}{0,0,.6}
\definecolor{dkyellow}{cmyk}{0,0,.8,.3}
\definecolor{frenchplum}{RGB}{129,20,83}
\lstset{
  language        = c,
  basicstyle      = \small\ttfamily,
  keywordstyle    = \bfseries\color{dkblue},
  stringstyle     = \color{red},
  morekeywords    = {assert, havoc, assume},
  commentstyle    = \color{dkgreen},
  escapeinside={@}{@},
  numberstyle=\tiny,
  numbers=left,
  numbersep=3pt,
  xleftmargin=1.1em,
  mathescape=true,  
  columns=flexible,
  classoffset=1, % starting new class
  morekeywords={invariant},
  keywordstyle=\bfseries\color{frenchplum},
  classoffset=0,
}

% defining Viper-like language for lstlisting
\lstdefinelanguage{Viper}{
  morekeywords={acc, method, struct,if,else,returns,procedure,requires,ensures,:=,var,
    new,old,free,implicit,modifies,call,locals,assume,assert,choose,havoc,ghost,
    predicate,function,invariant,while, return,atomic, split, type, field, result,
    mark, unmark, define, datatype, domain, axiom, forall, Int, Set},
  deletekeywords={union,int},
  % lineskip=-0.1em,
  numbers=left,
  xleftmargin=2em,
  escapeinside={@}{@},
  % numbers=none,
  % stepnumber=2,
  % firstnumber=1,
  % numberfirstline=true,
  numberstyle=\tiny,
  basicstyle=\footnotesize\ttfamily,
  columns=flexible,
  morecomment=*[s][\color{green!60!black}]{/*}{*/},
  morecomment=*[l][\color{green!60!black}]{//},
  moredelim=**[is][\color{purple}]{|<}{>|},
  mathescape=true,
}


\newcommand{\code}[1]{\textnormal{\texttt{#1}}}

% hyperref clashes with osdi20 template
% \usepackage[breaklinks=true,
%             pdfdisplaydoctitle=true,
%             %pagebackref, % we don't need this for a paper
%             bookmarksnumbered=true,
%             colorlinks = true,
%             citecolor = black, % change this if you want to highlight citations, maybe Blue
%             anchorcolor = black,
%             urlcolor = Blue, % apparently, Blue != blue
%             linkcolor = Blue,
%             pdfborder={0 0 0},
%             pdfpagelabels,
%             pdfpagelayout=SinglePage,
%             hyperfootnotes=false
%             ]{hyperref}

\usepackage[square,comma,numbers,sort&compress]{natbib}

%\usepackage{ulem}      % for strikethrough and underlining
%\mw{above package turns italics to underlines.}

\usepackage{textcomp}
%\usepackage{lastpage}
\usepackage{tabularx}
\usepackage{pifont}
%\usepackage{natbib}
\usepackage{url}
\def\UrlBreaks{\do\/\do-}
\usepackage{breakurl}

\usepackage{wrapfig}

\usepackage{ulem}
\normalem

% for cap the first letter for a cmd
\usepackage{mfirstuc}

% for getting table 1 in hotnets09 submission to work
\usepackage{colortbl} % colorful columns and rows
\usepackage{array}    % needed for 'b' argument in tabular preamble

\usepackage{dblfloatfix}

\usepackage[font=small]{caption}
\usepackage{subcaption}

%\usepackage{algorithm}
%\usepackage{algorithmic}
%\floatname{algorithm}{Pseudocode}
\usepackage[noend]{algpseudocode}
\algrenewcomment[1]{\hfill// #1}%
\algrenewcommand\algorithmicindent{1.25em}

\definecolor{darkgreen}{rgb}{0,0.75,0}

% Hack algpseudocode to be more Python-like
\algrenewcommand\algorithmicdo{:}
\algrenewcommand\algorithmicwhile{\textbf{while}}
\algrenewcommand\algorithmicfor{\textbf{for}}
\algrenewcommand\algorithmicforall{\textbf{for all}}
\algrenewcommand\algorithmicloop{\textbf{loop}}
\algrenewcommand\algorithmicrepeat{\textbf{repeat}}
\algrenewcommand\algorithmicuntil{\textbf{until}}
\algrenewcommand\algorithmicprocedure{\textbf{procedure}}
\algrenewcommand\algorithmicfunction{\textbf{function}}
\algrenewcommand\algorithmicif{\textbf{if}}
\algrenewcommand\algorithmicthen{:}
\algrenewcommand\algorithmicelse{\textbf{else}}
\algrenewcommand\algorithmicrequire{\textbf{Require}:}
\algrenewcommand\algorithmicensure{\textbf{Ensure}:}
\algrenewcommand\algorithmicreturn{\textbf{return}}
\algdef{SE}[WHILE]{While}{EndWhile}[1]{\algorithmicwhile\ #1\algorithmicdo}{\algorithmicend\ \algorithmicwhile}%
\algdef{SE}[FOR]{For}{EndFor}[1]{\algorithmicfor\ #1\algorithmicdo}{\algorithmicend\ \algorithmicfor}%
\algdef{S}[FOR]{ForAll}[1]{\algorithmicforall\ #1\algorithmicdo}%
\algdef{SE}[IF]{If}{EndIf}[1]{\algorithmicif\ #1\algorithmicthen}{\algorithmicend\ \algorithmicif}%
\algdef{C}[IF]{IF}{ElsIf}[1]{\algorithmicelse\ \algorithmicif\ #1\algorithmicthen}%
\algnotext{Function}                
\algnotext{EndFunction}
\algnotext{EndFor}
\algnotext{EndIf}
\algnotext{EndWhile}
\algnewcommand\Or{\textbf{or}\xspace}
\algnewcommand\myAnd{\textbf{and}\xspace}

\usepackage{amsmath,amscd}
\usepackage{amssymb}
\usepackage{amsfonts}
\usepackage{amsthm}
\usepackage{cleveref} % for clever references

\usepackage[noeka]{mathrmletter}
\usepackage{tgtermes}
%\renewcommand{\ttdefault}{cmtt}

% peanut gallery comments
% NOTE: Comment out the line below if you want a draft with no red comments.
% NOTE: Commenting out this line may replace some of the red comments with 
%       extra spaces or newlines.
%\def\noeditingmarks{}

\newcommand{\zdag}{{$^{\dagger}$}\xspace}
\newcommand{\zddag}{{$^{\ddagger}$}\xspace}
\newcommand{\zstar}{{$^{\star}$}\xspace}

\newcommand{\CF}[1]{\xmakefirstuc{#1}}

\newcommand{\sys}{Taster\xspace}
\newcommand{\Sys}{\CF{\sys}\xspace}
\newcommand{\merge}{\ensuremath{\code{merge}}\xspace}

\newcommand{\circledone}{\ding{192}\xspace}
\newcommand{\circledtwo}{\ding{193}\xspace}
\newcommand{\circledthree}{\ding{194}\xspace}
\newcommand{\circledfour}{\ding{195}\xspace}
\newcommand{\circledfive}{\ding{196}\xspace}
\newcommand{\filledone}{\ding{202}\xspace}
\newcommand{\filledtwo}{\ding{203}\xspace}
\newcommand{\filledthree}{\ding{204}\xspace}
\newcommand{\filledfour}{\ding{205}\xspace}
\newcommand{\filledfive}{\ding{206}\xspace}

\newcommand{\fillin}{\textred{fill in...}\xspace}

% MW: our "house style" is not to italicize "et al". (we also don't
% italicize "i.e." and "e.g.".) rationale: why draw attention to it?
% italics mean "from a foreign language", but is it necessary to
% emphasize the Latin-ness of "et al."?
\newcommand{\etal}{et al.\xspace}

\newcommand{\cpu}{\textsc{cpu}\xspace}

\makeatletter
\def\imod#1{\allowbreak\mkern10mu({\operator@font mod}\,\,#1)}
\makeatother

\makeatletter
\setlength{\@fptop}{0pt}
\makeatother

\newenvironment{theabstract}{\subsection*{Abstract}}{}

%\renewcommand{\vec}[1]{\mathbf{#1}}

%\def\compactify{\leftmargin=\parindent \itemsep=0.01pt \topsep=0.01pt \partopsep=0pt \parsep=0.01pt}
\def\compactify{\itemsep=0in \topsep=2pt \parsep=0.00in \partopsep=0pt
\leftmargin=2em}
\let\latexusecounter=\usecounter
\newenvironment{CompactItemize}
  {\def\usecounter{\compactify\latexusecounter}
   \begin{itemize}\addtolength{\itemsep}{-0.075in}}
  {\end{itemize}\let\usecounter=\latexusecounter}
\newenvironment{CompactEnumerate}
  {\def\usecounter{\compactify\latexusecounter}
   \begin{enumerate}}
  {\end{enumerate}\let\usecounter=\latexusecounter}

\newenvironment{myitemize}%
  {\begin{list}{\labelitemi}{\itemsep3pt \topsep3pt \parsep0.00in
  \partopsep=3pt \leftmargin1em}}%
  {\end{list}}
\newenvironment{myitemize2}%
  {\begin{list}{\labelitemi}{\itemsep1pt \topsep2pt \parsep0.00in
  \partopsep=0pt \leftmargin1.2em}}%
  {\end{list}}
\newenvironment{myitemize4}%
  {\begin{list}{\labelitemi}{\itemsep2pt \topsep2pt \parsep0.00in
  \partopsep=0pt \leftmargin1.2em}}%
  {\end{list}}
\newenvironment{myitemize5}%
  {\begin{list}{\labelitemi}{\itemsep3pt \topsep3pt \parsep0.00in
  \partopsep=3pt \leftmargin1.5em}}%
  {\end{list}}
\newenvironment{myitemize6}%
  {\begin{list}{\S3.3}{\itemsep3pt \topsep3pt \parsep0.00in
  \partopsep=3pt\leftmargin1em}}%
  {\end{list}}


%\newenvironment{myitemize}%
%  {\begin{list}{\labelitemi}{\itemsep4pt \topsep10pt \parsep0.00in
%  \partopsep=0pt}}%
%  {\end{list}}
\newenvironment{myenumerate}
  {\def\usecounter{\compactify\latexusecounter}
   \begin{enumerate}}
  {\end{enumerate}\let\usecounter=\latexusecounter}

\def\compactsortof{\itemsep=0in \topsep=2pt \parsep=0.00in \partopsep=0pt
\leftmargin=1.2em}
\newenvironment{myenumerate2}
  {\def\usecounter{\compactsortof\latexusecounter}
   \begin{enumerate}}
  {\end{enumerate}\let\usecounter=\latexusecounter}

%\def\compactsortof{\itemsep=3pt \topsep3pt \parsep=0ex \partopsep=0pt
%\leftmargin=1.55em}
\newenvironment{myenumerate3}
  {\def\usecounter{\compactsortof\latexusecounter}
   \begin{enumerate}}
  {\end{enumerate}\let\usecounter=\latexusecounter}

\def\compactsqueeze{\itemsep=0pt \topsep0pt \parsep=0ex \partopsep=0pt
\leftmargin=1.63em}
\newenvironment{myenumerate4}
  {\def\usecounter{\compactsqueeze\latexusecounter}
   \begin{enumerate}}
  {\end{enumerate}\let\usecounter=\latexusecounter}

%\ifextended
%\def\compactRenum{\itemsep=1ex \topsep=1ex \parsep=0.00in \partopsep=0pt
%\leftmargin=2.05em}
%\else
\def\compactRenum{\itemsep=0in \topsep=2pt \parsep=0.00in \partopsep=0pt
\leftmargin=2.05em}
%\fi
\newenvironment{myRenumerate}
    {\def\usecounter{\compactRenum\latexusecounter}\begin{enumerate}}
    {\end{enumerate}\let\usecounter=\latexusecounter}

\newcounter{saveenumi}

\newcommand{\astskip}{\smallskip\noindent\parbox{\linewidth}
			{\center*\hspace{2.5em}*\hspace{2.5em}*\medskip\smallskip}}

% uncomment to use regular paragraphs
%\def\normalpar{}

\ifx\normalpar\undefined
  \newcommand{\mypar}[1]{\textbf{#1}}
\else
  \newcommand{\mypar}[1]{\paragraph{#1}}
\fi

\def\discretionaryslash{\discretionary{/}{}{/}}
{\catcode`\/\active
\gdef\URLprepare{\catcode`\/\active\let/\discretionaryslash
        \def~{\char`\~}}}%
\def\URL{\bgroup\URLprepare\realURL}%
\def\realURL#1{\tt #1\egroup}%

\newcommand{\ttbox}[1]{\mbox{\texttt{#1}}\xspace}

\hyphenation{ra-tionale pseudo-constraint Face-dancer}


% circled numbers
\newcommand{\circled}[1]{\textcircled{{#1}}}
\newcommand{\term}[1]{{{#1}}}

\newcommand{\onarrow}[1]{\xrightarrow[]{\text{#1}}}

% common math short-hands

\newcommand{\m}[1]{\mathsf{#1}}
\newcommand{\tuple}[1]{\left\langle {#1} \right\rangle}
\newcommand{\setcomp}[2]{\ensuremath{\left\{#1\;\middle|\;#2\right\}}}
\newcommand{\stacklabel}[1]{\stackrel{\smash{\scriptscriptstyle \mathrm{#1}}}}
\newcommand{\Def}{\stacklabel{\mathrm{def}}}

\newcommand{\NN}{\mathbb{N}}
\newcommand{\ZZ}{\mathbb{Z}}

% macros related to examples
\newcommand{\LOAD}{\code{LOAD}}
\newcommand{\STORE}{\code{STORE}}
\newcommand{\IF}{\code{if}}

\newcommand{\lengthof}[1]{{#1.\code{len}}}
\newcommand{\mergeprog}{\code{merge}}
% FIND_MIN
\newcommand{\fmcount}{\ensuremath{\code{count}}\xspace}
\newcommand{\mina}{\ensuremath{\code{min}}\xspace}
% KMP
\newcommand{\txt}{\ensuremath{\code{text}}\xspace}
\newcommand{\pat}{\ensuremath{\code{pat}}\xspace}
\newcommand{\lps}{\ensuremath{\code{lps}}\xspace}
% 2D_CONVEX_HULL
\newcommand{\chstack}{\ensuremath{\code{stack}}\xspace}
\newcommand{\getx}[1]{{#1.\code{x}}}
\newcommand{\gety}[1]{{#1.\code{y}}}
\newcommand{\chinc}{\ensuremath{\code{in\_c}}\xspace}\newcommand{\chlast}{\ensuremath{\code{last}}\xspace}
\newcommand{\chnext}{\ensuremath{\code{next}}\xspace}

% macros for framework


\newcommand{\seq}[1]{\langle #1 \rangle}
\newcommand{\ts}{T}
\newcommand{\impl}{{I}}
\newcommand{\spec}{{S}}
\newcommand{\better}{B}
\newcommand{\refines}{\leq}
\newcommand{\tsspec}{\ts_\spec}
\newcommand{\tsimpl}{\ts_\impl}
\newcommand{\tsimplaux}{\ts_{\impl\better}}
\newcommand{\tsimplauxinv}{{\hat{\ts}}_{\impl\better}}
\newcommand{\tsbetter}{\ts_\better}
\newcommand{\transitions}{\Delta}
\newcommand{\locimpl}{\ell}
\newcommand{\locbetter}{\ell'}
\newcommand{\locspec}{\ell^\spec}
\newcommand{\observe}{\omicron}
\newcommand{\observations}{O}
\newcommand{\abstractfunc}{\alpha}
\newcommand{\noob}{\tau}
\newcommand{\actions}{A}
\newcommand{\action}{a}
\newcommand{\oaction}{b}
\newcommand{\states}{\Sigma}
\newcommand{\init}{\theta}
\newcommand{\inv}{\mathit{Inv}}
\newcommand{\state}{s}
\newcommand{\ostate}{s'}
\newcommand{\exec}{\rho}
\newcommand{\execsof}{\m{execs}}
\newcommand{\trace}{\sigma}
\newcommand{\otrace}{\sigma'}
\newcommand{\unstutter}{\sharp}
\newcommand{\traceof}{\m{trace}}
\newcommand{\tracesof}{\m{traces}}
\newcommand{\stepsto}[1][]{\ifstrempty{#1}{\rightarrow}{\xrightarrow[]{#1}}}
\newcommand{\movesto}[1]{\mathrel{\stackrel{#1}{\longrightarrow}\!\!{}^*}}
\newcommand{\lang}{\mathcal{L}}
\newcommand{\constraints}{\mathcal{C}}

\newcommand{\pc}{\code{pc}}
\newcommand{\at}[1]{{}@#1}
\newcommand{\retact}{\m{return}}
\newcommand{\callact}{\m{call}}

\newcommand{\precond}{\varphi_{\m{pre}}}
\newcommand{\postcond}{\varphi_{\m{post}}}
\newcommand{\codesub}[1]{${}_{\code{#1}}$}
% names

 
% Local Variables:
% tex-command: "gmake;:"
% tex-main-file: "icing.ltx"
% tex-dvi-view-command: "gmake preview;:"
% End:
